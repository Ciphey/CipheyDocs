%\documentclass[twocolumn]{article}
\documentclass{article}

\usepackage{amsmath}
\usepackage{amssymb}
\usepackage{amsthm}
\usepackage{graphicx}
\usepackage{cleveref}
\usepackage{enumerate}
\usepackage{gensymb}
\usepackage{geometry}
\usepackage{enumitem}
\graphicspath{{.}}

\newtheorem{theorem}{Theorem}
\newtheorem{lemma}[theorem]{Lemma}
\theoremstyle{definition}
\newtheorem{definition}{Definition}
\newtheorem{algorithm}{Algorithm}

\crefname{lemma}{Lemma}{Lemmas}
\crefname{theorem}{Theorem}{Theorems}
\crefname{definition}{Definition}{Definitions}

\title{AuSearch}
\author{Harlan Connor}

\begin{document}
\maketitle
\pagebreak

\section{Problem}
In Ciphey, we need to navigate a tree of unbounded depth, trying to find a 
solution node. Each node falls into the following types:
\begin{enumerate}[label=\Roman*]
\item Has zero or more children (a confirmable cipher)
\item Has one child which is definitely a solution, or no children (an inferable cipher)
\item  Has zero or more children, and can be done in neligible time (an encoding)
\end{enumerate}

Evaluating the children of a node is an intensive task, and so is checking nodes,
so we wish to minimise both actions. We must find such a node if there is any 
reasonable chance that it exists (above a threshold $\ell_p$), and if it can be 
found in reasonable time $\ell_t$.

To aid us in our quest, we have the following oracles for any node $A$:
\begin{itemize}
\item A metric that tells us either (for type I) the rough likelihood of there being any children ($p_A$), or (for type II) the rough likelihood of this being the final node ($P_A$); 
      the probability of a successful crack
\item The average running time of finding the children if there are some ($t_A$);
      the average runtime of a successful crack
\item The average running time of finding out that there are no children ($T_A$);
      the average running time of an unsuccessful crack
\end{itemize}

If we let $\omega_A$ be the time taken to run if $A$ is on the fastest route,
$\Omega_A$ be the time taken if it isn't, and $\tau_A$ be the time taken to examine the rest of the tree.
We can arrive at a rather simplistic optimal weight system right away:
\[
	W(A) = p_A \cdot (t_A + (1-P_A) (\omega_A + \tau_A)) + (1 - p_A) \cdot (T_A + \tau_A)
\]

Needless to say, $\tau_A$ is uncomputable without evaluating every node, as a
trivial result of the undecidability of the Halting Problem. $\omega_A$ and $\Omega_A$ have similar issues.

In this paper, we set about trying to find a replacement heuristic for $\tau_A$, as well as trying to approximate $\omega_A$ and $\Omega_A$.

We will use $\xi$ to represent the root node.

\section{Modelling}
%We will assume that all ciphers that are successfully cracked either yield the plaintext 
%directly, or are the ancestor of the plaintext. Even though this is not \emph{always}
%the case, we will be able to navigate out of such situations (at a time penalty).

We assume that there is no overhead in getting these stats. This is again untrue,
but if we require that any non-trivial work is done in the actual gathering of 
children, at the expense of weakening the heuristic.

We assume that type I nodes will have a probability of being the solution $P$ high
enough that we can treat it as 1 in the weight calculation. However, this is tweakable within the algorithm, so I will just leave $P$ as-is.

We also assume that there are no identical nodes. This is achieved in Ciphey by
filtering out reoccurring values, which causes reality to move away from the
heuristic a small bit.

We can assume that a successful crack means that we are on the right path.

%\section{Method}
%Early solutions did not keep track of what was below them, leading to a sunk cost
%fallacy, where the program would just recurse down the most efficient path at each 
%point.
%
%When discussing the problem with Brandon Skerritt, he briefly suggested Alpha-Beta 
%pruning and minimax style solutions to the problem. Thinking on this theme, I came upon
%this chess engine-like solution.
%
%Let's first examine a basic non-pruning chess engine. It has a tree of nodes, each node representing a
%position on the board. We can either heuristically evaluate it (how many pieces do we have,
%are they in strong positions), or we can look at the child nodes, evaluate them, and use the
%information to calculate a more informed evaluation of the position.
%
%This replacement of a single heuristic with a refined combination of heuristics only works because
%a position is only as good as where it gets you: an ideal engine would look at all the possible
%future positions that could arise, and work off that.
%
%Luckily for us, our problem can be thought of in a not dissimilar way.
%
%RANTS
%
%Each node $A$, when evaluated in some order $N$, either contains or does not contain
%the solution.
%
%We can evaluate the expected run time of a child as the expected run time of the root.
%
%We use the simplification that $\tau_\xi = \Omega_\xi$
%
%For type II nodes, $\omega_A = \Omega_A = 0$, and $p_A = P_A$, so $W(A) = (1-P_A)(T_A + \tau_A)$, a massive simplification!
%
%This isn't a tree, its a list of possible results!
%
%We do not care about the runtime if it is not a member of $L$
%
%We don't care about infinite recursion, as the heuristic now perfectly aims for
%what we want!
%
%	 Find some list $N$ formed of all of the elements of $L$,
%	      where each element only occurs once, which minimises 
%	      $p_{N_0}(t_{N_0} + \omega_{N_0}) + (1-p_{N_0})(p_{N_1}(t_{N_1} + \omega_{N_0}) + (1-p_{N_1})(...))$, which represents the average time taken if the solution is a member of $L$.
%	 Since, the probability that none of the nodes are the solution 
%	      $\prod_{i \in L} (1-P_L)$ is usually very small, we can approximate 
%	      $\omega_\xi$ as that value. 
%	 $\Omega_\xi$ could be infinite, but using the same assumption, we
%	      can approimate it as $\sum_{i \in N}(1 - p_i)(T_i + \Omega_i)$
%	      
%	      We dont care if type I or type II
%	      
%	      Time can be in any linear unit, so it doesn't matter if we measure on one machine
%
%	We assume $\forall A: \omega_A = \omega_\lambda \land \Omega_A = \Omega_\lambda $

\section{Proposed Solution}
Instead of storing the results as the intuitive tree data structure, it makes more
sense to think about it as a list, as we do not care what the parent of a node is!

We let Info($A$ : Node) be the function that gives us the information about the 
node, and Evaluate($A$ : Node) be the function that gives us the result of 
procesing.

We can gather $\omega_\xi$ and $\Omega_\xi$ through sampling for each type 
of data, and use them as an approximation for any $\omega_A$ and $\Omega_A$ that
we may encounter of that type.

\begin{algorithm}[FindBestOrder($S$ : List[Node])]\,\\
We need to find the node $A$ which has the lowest value of:
\[
	W(A) =  p_A \cdot t_A + (1 - p_A) \cdot (T_A + W(N))
\]

	Where $N$ is some a sequence formed from unique elements of $S \setminus \{A\}$, that minimises $W(A)$:
\[ 
	W(N) = 
	\begin{cases}
		0 & \text{if}\,|n| = 0 \\
		p_{N_0} t_{N_0} + (1 - p_{N_0}) (T_{N_0} + W(N')) & \text{otherwise}
	\end{cases}
\]

RETURN $<A, N>$
\end{algorithm}

\begin{algorithm}[CipheySearch(\textit{CText} : data)]\,\\
\begin{enumerate}
\item Let the set $L = \text{Evaluate}(<$Info(\textit{CText})$, [\,]>$
\item If \textit{CText} was the solution, return it.
\item Create our node pointer $A$
\item WHILE $|L| \neq 0$
	\begin{enumerate}
	\item Expand all encodings to their fullest depth
	\item If we have been running for more than $\ell_t$, ERROR "Timeout".
	\item $<A, N> := $ CALL FindBestOrder$(L)$
	\item Derive $p_N$ from $P_{[\,]} = 0$, $p_N := p_{N_0} + (1-p_{N_0})p_{N'}$
	\item If $p_N := p_{N_0} + (1-p_{N_0})p_{N'}$ is less than $\ell_p$, ERROR "Unlikely".
	\item Remove $A$ from $L$
	\item $S := \text{Evaluate}(A)$
	\item If it is the solution, RETURN $A$
	\item If it has no children, CONTINUE
	\item $L := L \cup \text{Info}(S)$
	\end{enumerate}
\item ERROR "List exhausted"
\end{enumerate}
\end{algorithm}
\end{document}
